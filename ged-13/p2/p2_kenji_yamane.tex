\documentclass{article}[twocolumn]
\usepackage[pdftex]{graphicx}
\usepackage[utf8]{inputenc}
\usepackage[brazil]{babel}
\usepackage{subfigure}
\usepackage{mathtools}
\usepackage{amsmath}
\usepackage{amssymb}
\usepackage{float}
\usepackage{tikz}

\title{RESOLUÇÃO DE PROVA: GED-13}
\author{Kenji Yamane}

\begin{document}
	\begin{figure}[H]
		\centering
		\includegraphics[width=10cm]{ita_logo.png}
	\end{figure}
	\begin{center}
		\textbf{\normalsize{INSTITUTO TECNOLÓGICO DE AERONÁUTICA}}\\
		\vspace{2.5cm}
		\large{Kenji de Souza Yamane}\\
		\vspace{4.7cm}
		\textbf{RESOLUÇÃO DA PROVA II}\\
		\large{GED 13 - PROBABILIDADE E ESTATÍSTICA}\\
		\vspace{4.7cm}
		\small{São José dos Campos – SP\\2020}
	\end{center}
	\newpage
	\tableofcontents
	\newpage
	\section{Introdução}
	O presente trabalho contém a solução da Prova II da matéria GED 13 - Probabilidade e
	Estatística do segundo ano do fundamental do ano de 2020.

	As resoluções quando necess\'ario se utilizam da linguagem de programação R, tendo
	como plataforma o \textit{Linux}, sobre linha de comando.

	As linhas de código foram diretamente copiadas da linha de comando do sistema operacional
	e inseridas em um código de \LaTeX, em ambiente \textit{verbatim}, sem nenhuma alteração.
	O código em \LaTeX, gerou então este arquivo. Todas as figuras deste relat\'orio foram
	geradas tamb\'em com \LaTeX, com o pacote \textit{tikz}.
	
	Como foram diretamente retirados da interface do \textit{Linux}, \textgreater é indicativo
	de comando, e + é indicativo de comando sequenciado para os c\'odigos em R.
	\newpage
	\section{Quest\~ao 1}
	Seja \textit{X} uma vari\'avel aleat\'oria com densidade dada por
	\begin{equation}
		f(x) = \left\{\begin{array}{lc}
			a(1 + x), & se \quad 0 < x \leq 1\\
			\frac{2}{3}, & se \quad 1 < x \leq 2 \\
			0, & $caso contr\'ario$
		\end{array}\right.
		\nonumber
	\end{equation}
	\subsection{a}
	Obtenha o valor de a.
	\begin{equation}
		\int_{-\infty}^{+\infty}f(x)dx = 1 \Rightarrow
		\int_{-\infty}^{0}f(x)dx + \int_{0}^{1}f(x)dx +
		\int_{1}^{2}f(x)dx + \int_{2}^{+\infty}f(x)dx = 1 \Rightarrow
		\nonumber
	\end{equation}
	\begin{equation}
		\Rightarrow \int_{0}^{1}f(x)dx + \int_{1}^{2}f(x)dx = 1 \Rightarrow
		\int_{0}^{1}a(1 + x)dx + \int_{1}^{2}\frac{2}{3}dx = 1 \Rightarrow
		\nonumber
	\end{equation}
	\begin{equation}
		\Rightarrow
		\int_{0}^{1}a(1 + x) = \frac{1}{3} \Rightarrow
		a(x + \frac{x^{2}}{2} + c) \big|_{0}^{1} = \frac{1}{3} \Rightarrow
		a(1 + \frac{1}{2}) = \frac{1}{3}\Rightarrow
		\nonumber
	\end{equation}
	\begin{equation}
		\Rightarrow
		a\frac{3}{2} = \frac{1}{3} \Rightarrow
		a = \frac{2}{9}
		\nonumber
	\end{equation}
	\subsection{b}
	Calcule $P\left(\frac{1}{2} \leq x \leq \frac{3}{2}\right)$.
	\begin{equation}
		P\left(\frac{1}{2} \leq x \leq \frac{3}{2}\right) =
		\int_{\frac{1}{2}}^{\frac{3}{2}}f(x)dx =
		\int_{\frac{1}{2}}^{1}f(x)dx + \int_{1}^{\frac{3}{2}}f(x)dx =
		\nonumber
	\end{equation}
	\begin{equation}
		= \int_{\frac{1}{2}}^{1}\frac{2}{9}(1 + x)dx + \int_{1}^{\frac{3}{2}}\frac{2}{3}dx
		= \frac{2}{9}(x + \frac{x^{2}}{2} + c)\big|_{\frac{1}{2}}^{1} +
		\frac{2}{3}(x + c)\big|_{1}^{\frac{3}{2}} =
		\nonumber
	\end{equation}
	\begin{equation}
		= \frac{2}{9}(\frac{1}{2} + \frac{3}{8}) + \frac{2}{3}\frac{1}{2} =
		\frac{2}{9}\frac{7}{8} + \frac{1}{3} = \frac{19}{36}
		\nonumber
	\end{equation}
	\newpage
	\section{Quest\~ao 2}
	Seja \textit{X} uma vari\'avel aleat\'oria discreta com fun\c{c}\~ao de probabilidade
	\begin{equation}
		p(x) = \frac{c}{4^{x}}, \quad x \in \mathbb{N}
		\nonumber
	\end{equation}
	\subsection{a}
	Obtenha o valor de \textit{c}.
	\begin{equation}
		\sum_{i}p_{i}(x) = 1 \Rightarrow \sum_{x = 0}^{\infty} \frac{c}{4^{x}} = 1 \Rightarrow
		\frac{c}{1 - \frac{1}{4}} = 1 \Rightarrow c\frac{4}{3} = 1 \Rightarrow c = \frac{3}{4}
		\nonumber
	\end{equation}
	\subsection{b}
	Obtenha a probabilidade de \textit{X} ser par.
	\begin{equation}
		P(X = 2k) = p(2k), \quad k \in \mathbb{N} \Rightarrow P(X = 2k) =
		\frac{c}{4^{2k}} \Rightarrow
		\nonumber
	\end{equation}
	\begin{equation}
		\Rightarrow
		P(X|2) = \sum_{k = 0}^{\infty} \frac{c}{16^k} = \frac{c}{1 - \frac{1}{16}} =
		\frac{3}{4}\frac{16}{15} = \frac{4}{5}
		\nonumber
	\end{equation}
	\newpage
	\section{Quest\~ao 3}
	Considere duas retas no $\mathbb{R}^{2}$, dadas por y = 0 e y = 1. Marylin
	est\'a parada no ponto de origem do plano. Ela escolhe um \^angulo $\theta$ distribu\'ido
	uniformemente em (0, $\pi$) e desenha um segmento de reta entre as linhas y = 0 e y = 1
	em um \^angulo $\theta$ a partir da origem no $\mathbb{R}^{2}$. Suponha que o segmento
	de reta encontre a linha y = 1 no ponto (X, 1).
	\begin{center}
		\begin{tikzpicture}
			\draw[thick, ->] (-2, 0) -- (2, 0) node[anchor=north west] {x};
			\draw[thick, ->] (0, -2) -- (0, 2) node[anchor=south east] {y};
			\foreach \x in {-1, 0, 1}
				\draw (\x cm, -1pt) -- (\x cm, 1pt);
			\foreach \y in {-1, 0, 1}
				\draw (-1 pt, \y cm) -- (1 pt, \y cm);
			\draw (-2, 1) -- (2, 1);
			\draw (0, 0) -- (1, 1) node[anchor=south] {\footnotesize (X, 1)};
			\filldraw [gray] (1, 1) circle (1pt);
			\draw (0.2cm, 0) arc (0:45:0.2cm);
			\node (A) at (0.35cm, 0.20cm) {$\theta$};
		\end{tikzpicture}
	\end{center}
	\subsection{a}
	Encontre a fun\c{c}\~ao de densidade de probabilidade de \textit{X}.

	Considere $\Theta$ a vari\'avel aleat\'oria associada \`a vari\'avel $\theta$.
	Assim:
	\begin{equation}
		P(\Theta \leq \theta) = \frac{\theta}{\pi}
		\nonumber
	\end{equation}
	Tem-se, portanto, que:
	\begin{equation}
		P(X \leq x) = P(\Theta \geq arccos\left(\frac{x}{\sqrt{x^{2} + 1}}\right)) =
		1 - P(\Theta \leq arccos\left(\frac{x}{\sqrt{x^{2} + 1}}\right)) =
		\nonumber
	\end{equation}
	\begin{equation}
		= 1 - \frac{arccos\left(\frac{x}{\sqrt{x^{2} + 1}}\right)}{\pi} \Rightarrow
		p(x) = \frac{d P(X \leq x)}{dx} = \frac{\frac{1}{\sqrt{1 - \frac{x^{2}}{x^{2} + 1}}}}{\pi}
		\frac{d\left[\frac{x}{\sqrt{x^{2} + 1}}\right]}{dx}
		\Rightarrow
		\nonumber
	\end{equation}
	\begin{equation}
		\Rightarrow p(x) = \frac{\sqrt{x^{2} + 1}}{\pi}\frac{1}{(x^{2} + 1)^{\frac{3}{2}}} =
		\frac{1}{\pi + \pi x^{2}}
		\nonumber
	\end{equation}
	Pode-se ainda confirmar que de fato ela constitui uma fun\c{c}\~ao densidade
	de probabilidade pois:
	\begin{equation}
		\int_{-\infty}^{+\infty}p(x)dx = \lim_{x \rightarrow +\infty}P(X \leq x) =
		1 - \lim_{x \rightarrow +\infty}\frac{arccos\left(\frac{x}{\sqrt{x^{2} + 1}}\right)}{\pi} =
		\nonumber
	\end{equation}
	\begin{equation}
		= 1 - \frac{arccos\left(\lim_{x \rightarrow \infty} \frac{1}{\sqrt{1 +
		\frac{1}{x^{2}}}}\right)}{\pi} =
		1 - 0 = 1
		\nonumber
	\end{equation}
	O gr\'afico correspondente \`a essa fdp \'e:
	\begin{center}
		\begin{tikzpicture}
			\draw[thick, ->] (-3, 0) -- (3, 0) node[right] {x};
			\draw[thick, ->] (0, -1.5) -- (0, 1.5) node[right] {y};
			\draw (-1 pt, 1 cm) -- (1 pt, 1 cm) node[anchor=west] {1};
			\draw[scale=1, domain=-3:3, smooth, variable=\x, blue]
			plot ({\x}, {1/(pi + pi*\x*\x)});
		\end{tikzpicture}
	\end{center}
	\subsection{b}
	Encontre o valor esperado [esperan\c{c}a] da densidade obtida em (a).
	\begin{equation}
		E[X] = \int_{-\infty}^{+\infty}xp(x)dx = \int_{-\infty}^{+\infty}\frac{x}{\pi + \pi x^{2}}dx
		= \int_{-\infty}^{0}\frac{x}{\pi + \pi x^{2}}dx +
		\int_{0}^{+\infty}\frac{x}{\pi + \pi x^{2}}dx
		\nonumber
	\end{equation}
	Analisando uma das parcelas:
	\begin{equation}
		\int_{-\infty}^{0}\frac{x}{\pi + \pi x^{2}}dx =
		\frac{ln(\pi)}{2\pi} -
		\lim_{x \rightarrow -\infty} \frac{ln(\pi + \pi x^{2})}{2\pi}
		\nonumber
	\end{equation}
	O qual diverge. Portanto, a esperan\c{c}a requisitada n\~ao existe.
	\section{Quest\~ao 4}
	O relatório de uma empresa de seguros apresenta dados estatísticos de danos (sinistros)
	de veículos na cidade de São Paulo. Os dados mostram que os quilômetros anuais do
	veículo (ou seja, quilômetros por veículo por ano) dirigidos entre os sinistros de trânsito
	podem ser representados por uma variável aleatória Normal com média de 15000
	quilômetros por ano e um coeficiente de variação \footnote{Note que est\'a escrito coeficiente
	de varia\c{c}\~ao, e n\~ao vari\^ancia.}
	de 25\%.
	\begin{equation}
		CV = 100\frac{s}{\bar{x}} \Rightarrow \frac{s}{\bar{x}} = 0.25 \Rightarrow
		s = \frac{15000}{4} = 3750
	\end{equation}
	\subsection{a}
	Qual é a probabilidade de um motorista (típico) que dirige 10000 quilômetros por ano
	sofrer um sinistro em um ano?

	Se X for maior que 10000, ele n\~ao sofrer\'a nenhum sinistro. Se for menor que 5000, sofrer\'a
	mais do que um sinistro. Portanto, o valor requisitado corresponde \`a:
	\begin{equation}
		P(5000 \leq X \leq 10000) = P(X \leq 10000) - P(X \leq 5000)
		\nonumber
	\end{equation}
	Utilizando o R:
	\begin{verbatim}
		> pnorm(10000, mean = 15000, sd = 3750) - pnorm(5000, mean = 15000, sd = 3750)
		[1] 0.08738084
	\end{verbatim}
	Portanto, a probabilidade \'e de 8,74\%.
	\subsection{b}
	Se o motorista típico rodou 8000 quilômetros em um determinado ano sem ter se
	acidentado, qual a probabilidade de ele se acidentar no restante do ano?
	
	Corresponde \`a:
	\begin{equation}
		P(X \geq 8000) = 1 - P(X \leq 8000)
		\nonumber
	\end{equation}
	Usando o R:
	\begin{verbatim}
		> 1 - pnorm(8000, mean = 15000, sd = 3750)
		[1] 0.9690259
	\end{verbatim}
	Portanto a probabilidade \'e de 96,9\%.
\end{document}
